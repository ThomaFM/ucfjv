\chapter{Graph}

\section{Fundamentals}
	\kactlimport{TopoSort.h}
	\kactlimport{dsu.h}

\section{Network flow}
	\kactlimport{Dinic.h}

\section{Math}
	\subsection{Number of Spanning Trees}
		% I.e. matrix-tree theorem.
		% Source: https://en.wikipedia.org/wiki/Kirchhoff%27s_theorem
		% Test: stress-tests/graph/matrix-tree.cpp
		Create an $N\times N$ matrix \texttt{mat}, and for each edge $a \rightarrow b \in G$, do
		\texttt{mat[a][b]--, mat[b][b]++} (and \texttt{mat[b][a]--, mat[a][a]++} if $G$ is undirected).
		Remove the $i$th row and column and take the determinant; this yields the number of directed spanning trees rooted at $i$
		(if $G$ is undirected, remove any row/column).

	\subsection{Erdős–Gallai theorem}
		% Source: https://en.wikipedia.org/wiki/Erd%C5%91s%E2%80%93Gallai_theorem
		% Test: stress-tests/graph/erdos-gallai.cpp
		A simple graph with node degrees $d_1 \ge \dots \ge d_n$ exists iff $d_1 + \dots + d_n$ is even and for every $k = 1\dots n$,
		\[ \sum _{i=1}^{k}d_{i}\leq k(k-1)+\sum _{i=k+1}^{n}\min(d_{i},k). \]
